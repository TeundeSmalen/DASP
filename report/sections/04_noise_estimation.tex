\section{Noise PSD Estimation} \label{sec:noise_estimation}
For the noise estimation, it is considered that speech and noise are independent and uncorrelated. From this Equations \ref{eq:H0} and \ref{eq:H1} can be derived. Whether $H_0$ or $H_1$ corresponds to a particular timeframe is determined by the Voice Activity Detector in Section \ref{sec:vad}.
\begin{align}
  H_{0}: & Y_{K}(l) = N_{k}(l) \quad \text{(speech absence)}
  \label{eq:H0} \\
  H_{1}: & Y_{K}(l) = S_{k}(l) + N_{k}(l) \quad \text{(speech presence)}
  \label{eq:H1}
\end{align}

\subsection{Voice Activity Dectector}
The method used to estimate the noise PSD is based on Voice Activity Detector. This Noise PSD estimation is given as Equation \ref{eq:smoothingsigma}. Where $\alpha$ is the smoothing constant. As described in \cite{Hendriks}, this method works well for noise with low variation in time.

\begin{equation}
  \sigma_{N,k}^{2}(l)=
  \begin{dcases}
      \alpha \hat \sigma_{N,k}^{2}(l-1) + (1-\alpha)\abs{y_{k}(l)}^{2} & \text{when } H_{0}(l)\\
      \hat \sigma_{N,k}^{2}(l-1) & \text{when } H_{1}(l)
  \end{dcases}
  \label{eq:smoothingsigma}
\end{equation}

\todo[inline]{Noise PSD vergelijken}
%Assuming the VAD used as described in Section \ref{sec:vad}, the following noise estimation is obtained.
%This method is 

\subsection{Minimum Statistics Method}
The Minimum Statistics Method does not depend on the Voice Activity Detector and only depends on previous time frames of $Y_{l}$ as can be seen in Equation \ref{eq:Q} and \ref{eq:Qmin}.
\begin{align}
  \mathbf{Q} &=
  \begin{Bmatrix}
    P_{YY,k}(l-M+1) & \hdots & P_{YY,k}(l)
  \end{Bmatrix}
  \label{eq:Q} \\
  \hat \sigma_{N,k}^{2}(l) &= Q_{min}
  \label{eq:Qmin}
\end{align}

\todo[inline]{Noise PSD vergelijken}

\subsection{MMSE Method}
One of the more extensive methods discussed during the course is the MMSE Method. Equation \ref{eq:noiseMMSE1} looks similar to Equation \ref{eq:smoothingsigma} when speech is not present. Instead of using the PSD of the input signal $y_k(l)$, the noise its expected value is determined for that specific time-frame. This can be done as in Equation \ref{eq:noiseMMSE2}.
\begin{align}
  \widehat{\sigma_{N}^{2}}(l) &= \alpha \widehat{\sigma_{N}^{2}}(l-1) + (1-\alpha)E\left[ \abs{N_{k}(l)}^{2}\abs{y_{k}(l)}\right]
  \label{eq:noiseMMSE1} \\
  E\left[ \abs{N_{k}(l)}^{2}\abs{y_{k}(l)}\right] &=
  P\left(H_{0,k}(l)|y_{k}(l)\right) E\left[\abs{N_{k}(l)}^{2}\abs{y_{k}(l)},H_{0,k}\right] +
  P\left(H_{1,k}(l)|y_{k}(l)\right) E\left[\abs{N_{k}(l)}^{2}\abs{y_{k}(l)},H_{1,k}\right]
  \label{eq:noiseMMSE2}
\end{align}

There are some unknowns present. The change that speech is absent can be rewritten as in Equation \ref{eq:MMSEu1}. The expected values can be given as Formulas \ref{eq:MMSEu2} and \ref{eq:MMSEu3}.
\begin{align}
  P\left(H_{0,k}(l)|y_{k}(l)\right) &= 1 - P\left(H_{1,k}(l)|y_{k}(l)\right)
  \label{eq:MMSEu1} \\
  E\left[\abs{N_{k}(l)}^{2}\abs{y_{k}(l)},H_{0,k}\right] &= \abs{y_{k}(l)^{2}}
  \label{eq:MMSEu2} \\
  E\left[\abs{N_{k}(l)}^{2}\abs{y_{k}(l)},H_{1,k}\right] &= \widehat{\sigma_{N}^{2}}(l-1)
  \label{eq:MMSEu3}
\end{align}

This leaves only one function to be determined as which can be further broken down as in Equation \ref{eq:MMSEu4}. Of which the two elements are given by Equations \ref{eq:pyh0} and \ref{eq:pyh1}.
\begin{align}
  P\left(H_{1,k}(l)|y_{k}(l)\right) &= \frac{P\left(H_{1,k}(l))\right)p_{Y|H_{1}}}{P\left(H_{1,k}(l))\right)p_{Y|H_{1}} + P\left(H_{0,k}(l))\right)p_{Y|H_{0}}}
  \label{eq:MMSEu4}\\
  p_{Y|H_{0}} &= \frac{1}{\widehat{\sigma_{N}^{2}} \pi} \exp\left(-\frac{\abs{y^{2}}}{\widehat{\sigma_{N}^{2}}}\right)
  \label{eq:pyh0}\\
  p_{Y|H_{0}} &= \frac{1}{\widehat{\sigma_{N}^{2}} (1+\xi_{H_{1}})\pi} \exp\left(-\frac{\abs{y^{2}}}{\widehat{\sigma_{N}^{2}}(1+\xi_{H_{1}})}\right)
  \label{eq:pyh1}
\end{align}

However, $\xi_{H_{1}}$ is unknown and there is no sufficient a priori knowledge present to give an estimation of this variable. Which makes this method unfeasible to use.