\section{SNR Estimation} \label{sec:snr_estimation}
The SNR estimation is used for a few subsystems. The Voice Ativity Detection in Section \ref{sec:vad} and the Target Estimation for the Wiener filter in Section \ref{sec:target_estimation}. There are two main approaches to estimate the SNR, these are discussed in this Section.
First the definition of the SNR is written down in Eq. \ref{eq:SNR}. The SNR is the power of the signal divided by the power of the noise. Often, this is expressed in decibel.

\begin{equation}
  \xi = \frac{\sigma_{S,k}(l)^{2}}{\sigma_{N,k}(l)^{2}} =
  \frac{P_{SS,k}}{P_{NN,k}} =
  \frac{E\left\{\abs{S_{k}(l)}^{2}\right\}}{E\left\{\abs{N_{k}(l)}^{2}\right\}}
  \label{eq:SNR}
\end{equation}

\subsection{Maximum Likelihood}
The first estimation approach is based on the maximum likelihood. Here, the signal power is estimated by the Bartlett estimate of the input power. The expected value of the noise power is divided by the order of the Bartlett estimate to use as the denominator.
\begin{align}
  \xi_{k}(l) &= \frac{E\left\{\abs{Y_{k}(l)}^{2}\right\}}{E\left\{\abs{N_{k}(l)}^{2}\right\}} - 1
  \label{eq:ML1}\\
  &= \frac{\hat P_{YY,k}^{B}(l)}{\frac{1}{L}E\left\{\abs{N_{k}(l)}^{2}\right\}}
  \label{eq:M2}
\end{align}

\subsection{Decision Directed}
The second estimation approach is more practical. 
\begin{align}
  \xi_{k}(l) &= \alpha \frac{E\left\{\abs{S_{k}(l)}^{2}\right\}}{E\left\{\abs{N_{k}(l)}^{2}\right\}} +
  (1-\alpha)\left(\frac{E\left\{\abs{Y_{k}(l)}^{2}\right\}}{E\left\{\abs{N_{k}(l)}^{2}\right\}} - 1\right)
  \label{eq:dd1}\\
  \abs{S_{k}(l)}^{2} &= \abs{\hat S_{k}(l-1)}^{2}
  \label{eq:dd2} \\
  \frac{E\left\{\abs{Y_{k}(l)}^{2}\right\}}{E\left\{\abs{N_{k}(l)}^{2}\right\}} - 1 &=
  \max \left[(\frac{\abs{Y_{k}(l)}^{2}}{E\left\{\abs{N_{k}(l)}^{2}\right\}}-1,0)\right]
\end{align}
